\documentclass[12pt,utf8,notheorems,compress]{beamer}

\usepackage[english]{babel}

\usepackage{mathtools}
\usepackage{booktabs}
\usepackage{ragged2e}
\usepackage{multicol}
\usepackage{tabto}

\usepackage[protrusion=true,expansion=false]{microtype}

\setlength\parskip{\medskipamount}
\setlength\parindent{0pt}

\renewcommand{\U}{\mathcal{U}}
\newcommand{\RR}{\mathbb{R}}
\newcommand{\NN}{\mathbb{N}}
\newcommand{\Id}{\mathrm{Id}}
\newcommand{\fst}{\mathsf{fst}}
\newcommand{\snd}{\mathsf{snd}}
\newcommand{\refl}{\mathsf{refl}}
\renewcommand{\succ}{\mathsf{succ}}
\newcommand{\seg}{\mathsf{seg}}
\newcommand{\base}{\mathsf{base}}
\newcommand{\lloop}{\mathsf{loop}}
\newcommand{\surf}{\mathsf{surf}}
\newcommand{\merid}{\mathsf{merid}}
\newcommand{\N}{\mathsf{N}}
\renewcommand{\S}{\mathsf{S}}
\newcommand{\IsContr}{\mathsf{IsContr}}
\newcommand{\IsProp}{\mathsf{IsProp}}
\newcommand{\IsSet}{\mathsf{IsSet}}
\newcommand{\fib}{\mathsf{fib}}
\newcommand{\defeq}{\vcentcolon=}
\newcommand{\defeqv}{\vcentcolon\equiv}

\title{Homotopy type theory}
\author[Arbeitsseminar Geometrie/Topologie]{\vspace{-1em}\\\includegraphics[scale=0.3]{torus.png} \\[0.5em] Ingo Blechschmidt \\[-0.3em] {\scriptsize November 25th, 2014}}
\date{November 25th, 2014}

\usetheme{Warsaw}
\usecolortheme{seahorse}
%\usefonttheme{default}
\usepackage{kurier}
\useinnertheme{rectangles}

\setbeamertemplate{frametitle}[default][colsep=-4bp,rounded=false,shadow=false,center]

\setbeamertemplate{headline}{}
\setbeamertemplate{navigation symbols}{}

\newcommand{\backupstart}{
  \newcounter{framenumberpreappendix}
  \setcounter{framenumberpreappendix}{\value{framenumber}}
}
\newcommand{\backupend}{
  \addtocounter{framenumberpreappendix}{-\value{framenumber}}
  \addtocounter{framenumber}{\value{framenumberpreappendix}} 
}

\newcommand*\oldmacro{}%
\let\oldmacro\insertshorttitle%
\renewcommand*\insertshorttitle{%
  \oldmacro\hfill\insertframenumber\,/\,\inserttotalframenumber\hfill}

\newcommand{\hil}[1]{{\usebeamercolor[fg]{item}{\textbf{#1}}}}

\newcommand{\img}[3]{\begin{center}\includegraphics[scale=#1]{#2}\\\small#3\end{center}}
%\newcommand{\imageslide}[3]{\frame{\frametitle{#1}\img{#2}{#3}}}

\setbeameroption{show notes}
\setbeamertemplate{note page}[plain]

\begin{document}

\frame{\titlepage}

\frame[t]{\frametitle{Outline}\tableofcontents}

\note{\justifying\fontsize{8pt}{9.6}\selectfont
  \setlength{\columnsep}{1em}
  \begin{multicols}{2}
    Homotopy type theory is a new branch of mathematics that combines
    aspects of several different fields in a surprising way. It is part of
    Voevodsky's \emph{univalent foundations} program and based on a recently
    discovered connection between homotopy theory and type theory, a branch
    of mathematical logic and theoretical computer science.

    In homotopy type theory, any set (really: \emph{type}) behaves like a
    topological space, or more precisely, a homotopy type. The basic notion
    of equality is reimagined in an interesting way: Analogous to how two
    given points in a space may be joined by more than one path, two
    elements of a set can be equal in many ways. A new axiom, the
    \emph{univalence axiom}, posits that equivalent structures really are
    the same, thus formalizing a widespread notational practice.

    Besides explaining how working in homotopy type theory feels like, the
    talk will give answers to the listed questions.
    The talk does not assume any background in formal logic or type theory.
    \columnbreak

    \begin{itemize}
    \justifying
    \item What are logical foundations for mathematics and why should we care?
    \item What are the disadvantages of traditional set-based approaches to foundations?
    \item Why is the development of homotopy theory radically simplified in
    homotopy type theory?
    \item How are the seemingly diverse activities of \emph{proving
    propositions} and \emph{exhibiting constructions} identified?
    \item How do inductive definitions of important spaces concisely capture
    their homotopy-theoretic content?
    \item Why is homotopy type theory a major step towards practically useful
    and easily applicable proof assistants?
    \end{itemize}
  \end{multicols}
}


\section{Foundations}

\subsection{What are foundations?}

\frame[t]{\frametitle{What are foundations?}
  \begin{itemize}
    \item Foundations set the logical context for doing maths.
    \item Their details don't matter in everyday work (mostly).
    \item But their main concepts do.
  \end{itemize}

  \pause
  \begin{itemize}
    \item Classical foundations are \emph{set-based} (ZF, ZFC, \ldots):
          \hil{Everything is a set.}
    \item $0 \defeq \emptyset$, \quad
          $1 \defeq \{0\}$, \quad
          $2 \defeq \{0,1\}$, \quad
          $\ldots$
    \item $(x,y) \defeq \{ \{x\}, \{x,y\} \}$ \quad (Kuratowski pairing)
    \item $(x,y,z) \defeq (x,(y,z))$
    \item maps: $(X,Y,R)$ with $R \subseteq X \times Y$ such that \ldots
  \end{itemize}
}

\note{
  \begin{itemize}
    \item Foundations allow us to be maximally precise.
    \item A \emph{proof} as commonly understood is really a shorthand for a
    (never spelled out) fully formal proof.
    \item Unlike informal proofs, the correctness of a formal proof can be
    checked mechanically.
  \end{itemize}

  \img{0.5}{logicomix}{Logicomix: An Epic Search for Truth}
}


\subsection{What's problematic with set-based foundations?}

\frame[t]{\frametitle{What's wrong with set-based foundations?}
  Set-based foundations \ldots
  \begin{itemize}
    \item allow to formulate nonsensical questions,
    \item do not reflect typed mathematical practice,
    \item require complex encoding of ``higher-level'' subjects,
          complicating interactive proof environments.
  \end{itemize}
}

\note{
  \begin{itemize}
    \item Examples for questions which can be formulated:
    \begin{itemize}
      \item Is $2 = (0,0)$? (No, when using my definitions.)
      \item Is $\sin \in \pi$? (Depends on your definitions.)
    \end{itemize}
    \item\justifying In ordinary practice, these questions would be deemed as nonsensical,
    since they disrespect the \emph{types} of mathematical objects and are not
    invariant under isomorphisms of the involved structures.
    \item Note: There are also \emph{structural approaches} to set theory
    without a global membership predicate (e.\,g.\@ ETCS), resolving this
    defect.
  \end{itemize}
}

\note{
  \begin{itemize}
    \item\justifying Fully unravel the definition of ``manifold'' in set-theoretical
    language to get a grasp of the complex encodings needed.
    \item This is no problem for humans, but it is for machines.
    \item Voevodsky: ``The roadblock that prevented generations of interested
    mathematicians and computer scientists from solving the problem of computer
    verification of mathematical reasoning was the unpreparedness of
    foundations of mathematics for the requirements of this task.''
  \end{itemize}

  \begin{itemize}
    \item Note: Set theory is perfectly fine for studying \emph{sets}.
  \end{itemize}
}


\section{Basics on homotopy type theory (HoTT)}

\subsection{What is homotopy type theory?}

\frame[t]{\frametitle{What is homotopy type theory?}
  \begin{itemize}
    \item Homotopy type theory is a new foundational theory.
    \item Basic notions have a homotopy-theoretic flavour.
    \item One can start doing ``real mathematics'' right away, without complex encodings.
    \item Initiated by Voevodsky in 2005.
  \end{itemize}

  \img{0.2}{authors}{Some participants of the IAS special year}
}

\note{
  \begin{itemize}
    \item\justifying Homotopy type theory is approximately intensional Martin-Löf type
    theory (existing since the 1970s) plus the new \emph{univalence axiom}.
    \item After repeatedly experiencing mistakes in his field going unnoticed
    for several years, Voevodsky wanted to work with proof assistants. He went
    public in 2009.
  \end{itemize}
}

\subsection{What are values and types?}

\frame[t]{\frametitle{What are values and types?}
  \begin{itemize}
    \item In type theory, there are \hil{values} and \hil{types}.
    \item Every value is of exactly one type.
    \item Types may depend on values.
  \end{itemize}
  \begin{align*}
    7 &: \NN \\
    (3,5) &: \NN \times \NN \\
    \succ &: \NN \to \NN \\
    \text{zero vector} &: \RR^n \quad\text{($n : \NN$)}
  \end{align*}

  \pause
  Let~$B(x)$ be a type family depending on $x:A$.
  \begin{itemize}
    \item $\sum_{x:A} B(x) = \text{``$\{ (a,b) \,|\, a:A, b:B(a) \}$''}$
    \item $\prod_{x:A} B(x) = \text{``$\{ f : A \to {??} \,|\, \text{$f(a) : B(a)$ for
    all $a:A$} \}$''}$
  \end{itemize}
}

\note{
  \begin{itemize}
    \item\justifying Types are familiar from programming (\texttt{Int},
    \texttt{String}, \ldots).
    \item But the type systems of well-known mainstream languages are either
    trivial (Ruby, Python: everything is an object) or not very expressive (C,
    Java).
    \item Haskell and languages of the ML family have a rich type system,
    encompassing function types and algebraic data types.
    \item But even their type systems do not support \emph{dependent types} --
    types which may depend on values. Look to Coq or Agda for those.
  \end{itemize}
}

\note{
  In the special case that~$B(x) \defeqv B$ does not depend on $x$:
  \[ \sum_{x:A} B \equiv A \times B \qquad
    \prod_{x:A} B \equiv (A \to B) \]
}


\subsection{What is the dependent equality type?}

\frame[t]{\frametitle{What is the dependent equality type?}
  In set theory, for a set~$X$ and elements~$x,y \in X$:
  \begin{itemize}
    \item ``$x=y$'' is a \hil{proposition}.
    \item Set theory is \hil{layered above} predicate logic.
  \end{itemize}
  \bigskip

  In intens.\@ type theory, for a type~$X$ and values~$x,y : X$:
  \begin{itemize}
    \item There is the \hil{equality type} $\Id_X(x,y)$ or $(x =_X y)$.
    \item To verify that ``$x=y$'',
    exhibit a value of~$(x = y)$.
    \item Have $\refl_x : (x = x)$.
    \item Identity types may contain zero or \hil{many} values!
  \end{itemize}
  Intuition: $(x = y)$ is the type of \hil{proofs} that ``$x=y$''.

  \pause
  Intuition: $(x = y)$ is the type of \hil{paths} $x \leadsto y$.
}

\note{
  \begin{itemize}
    \item\justifying Note that we use logical terminology. A proposition is merely a
    statement, not necessarily a true statement.
    \item In an intensional type theory, propositions are not an extra part of
    the language, distinct from values and types.
    \item Instead, \emph{propositions are (some) types}.
    \item To prove a proposition means to exhibit a value of it.
    Such a value can be thought of as a \emph{proof} or \emph{witness}.
    \item We have \emph{proof relevance}.
    \item (Not all types are propositions, see below for $\IsProp$.)
  \end{itemize}
}

\note{
  Examples for more complex propositions (types):
  \begin{itemize}
    \item ``$X$ is a subsingleton'': \tabto{4.7cm}
    $\prod_{x:X} \prod_{y:X} (x=y)$
    \item ``Addition is commutative'': \tabto{4.7cm}
    $\prod_{n:\NN} \prod_{m:\NN} (n+m = m+n)$
    \item ``Every number is even'': \tabto{4.7cm}
    $\prod_{n:\NN} \sum_{m:\NN} (n=2m)$
  \end{itemize}
}


\section{Homotopy theory in HoTT}

\subsection{How are types like spaces?}

\frame[t]{\frametitle{How are types like spaces?}
  \begin{center}\begin{tabular}{ll}
    \toprule
    homotopy theory & type theory \\\midrule
    \hil{space} $X$ & type $X$ \\
    \hil{point} $x \in X$ & value $x:X$ \\
    \hil{path} $x \leadsto y$ & value of $(x = y)$ \\
    \hil{(continuous) map} & value of $X \to Y$ \\
    \bottomrule
  \end{tabular}\end{center}

  \begin{itemize}
    \item A \hil{homotopy} between maps $f, g : X \to Y$ is a value of
    \[ (f \simeq g) \defeqv \prod_{x:X} (f(x) = g(x)). \]
    \item A space $X$ is \hil{contractible} iff
    \[ \IsContr(X) \defeqv \sum_{x:X} \prod_{y:X} (x=y). \]
  \end{itemize}
}

\frame[t]{\frametitle{How are types like spaces?}
  \begin{itemize}
  \item ``The type $X$ is \hil{contractible}'':
    \[ \IsContr(X) \defeqv \sum_{x:X} \prod_{y:X} (x=y). \]

  \item ``The type $X$ is a \hil{mere proposition}'':
  \[ \IsProp(X) \defeqv \prod_{x,y:X} \IsContr(x = y) \]

  \item ``The type $X$ is a \hil{set} or \hil{discrete space}'':
  \[ \IsSet(X) \defeqv \prod_{x,y:X} \IsProp(x = y) \]

  \item For instance, $\NN$ is a set.
  \end{itemize}
}


\subsection{How are constructions encoded?}

\frame[t]{\frametitle{How are constructions encoded?}
  \begin{itemize}
    \item The \hil{fiber} of a map $f : X \to Y$ over a point $y:Y$ is
    \[ \fib_f(y) \defeqv \sum_{x:X} (f(x) = y). \]

    \item The \hil{path space} of $X$ is
    \[ X^I \defeqv \sum_{x,y:X} (x=y). \]

    \item The \hil{based loop space} of $X$ at $x$ is
    \[ \Omega^1(X,x) \defeqv (x=x). \]

    \item The \hil{path fibration} of $(X,x)$ is the map
    \[ \snd : \sum_{y:X} (x=y) \to X. \]
  \end{itemize}
}


\subsection{What are higher inductive definitions?}

\frame[t]{\frametitle{What are higher inductive definitions?}
  The type~$\NN$ of natural numbers is \hil{freely generated} by
  \begin{itemize}
    \item a point $0 : \NN$ and
    \item a function $\succ : \NN \to \NN$.
  \end{itemize}
  This definition gives rise to an \hil{induction scheme}
  \[ \prod_{A:\NN \to \U}
    \Bigl(
    A(0) \times \Bigl(\prod_{n:\NN} A(n) \to A(\succ(n))\Bigr) \longrightarrow
    \prod_{n:\NN} A(n)
    \Bigr), \]
  and a \hil{recursion scheme}
  \[ \prod_{X:\U}
    \Bigl(
    X \times \Bigl(\NN \to (X \to X)\Bigr) \longrightarrow
    (\NN \to X)
    \Bigr). \]
}

\note{
  \begin{itemize}
    \item\justifying $\U$ is a \emph{universe}. Its values are types.
    \item The recursion scheme is the specialization of the induction scheme to
    constant type families $A(n) \equiv X$.
    \item In a \emph{higher} inductive definition, constructors may not only
    generate \emph{points}, but also \emph{paths} and \emph{higher paths}. 
  \end{itemize}
}

\frame[t]{\frametitle{How to present famous spaces?}
  The \hil{interval} $I$ is freely generated by
  \begin{itemize}
    \item a point $0 : I$ and
    \item a point $1 : I$ and
    \item a path $\seg : (0 = 1)$.
  \end{itemize}
  \bigskip

  The \hil{circle} $S^1$ is freely generated by
  \begin{itemize}
    \item a point $\base : S^1$ and
    \item a path $\lloop : (\base = \base)$.
  \end{itemize}
  \bigskip

  The \hil{sphere} $S^2$ is freely generated by
  \begin{itemize}
    \item a point $\base : S^2$ and
    \item a path $\surf : (\refl_\base = \refl_\base)$.
  \end{itemize}
}

\frame[t]{\frametitle{How to present famous spaces?}
  The \hil{suspension} $\Sigma X$ of $X$ is freely generated by
  \begin{itemize}
    \item a point $\N : \Sigma X$ and
    \item a point $\S : \Sigma X$ and
    \item a function $\merid : X \to (N = S)$.
  \end{itemize}
}


\section{References}

\frame[t]{\frametitle{References}
  \begin{itemize}
    \item the book (XXX)

    \item Voevodsky on his motivations:

    \url{http://www.math.ias.edu/~vladimir/Site3/Univalent_Foundations_files/2014_IAS.pdf}

    \item \url{http://www.math.ias.edu/~mshulman/hottseminar2012/01intro.pdf}
  \end{itemize}
}

\backupstart
\backupend

\end{document}

XXX: Illustration of sets and types
